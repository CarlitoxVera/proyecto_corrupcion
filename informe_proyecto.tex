\documentclass[a4paper, 12pt]{article}
\usepackage[utf8]{inputenc}
\usepackage[spanish]{babel}
\usepackage{geometry}
\usepackage{lipsum}
\usepackage{hyperref}
\geometry{margin=1in}

\title{\textbf{Informe del Proyecto: Análisis de Corrupción en Chile}}
\author{Equipo de Trabajo}
\date{Enero 2025}

\begin{document}

\maketitle

\section*{Introducción}
Este informe detalla el desarrollo, implementación y análisis del proyecto que busca modelar y analizar las relaciones entre personas, instituciones y empresas implicadas en casos de corrupción en Chile, particularmente aquellos revelados por el audio del abogado Luis Hermosilla.

\section*{Objetivos del Proyecto}
\begin{itemize}
    \item Proporcionar herramientas para analizar conexiones entre individuos e instituciones en casos de corrupción.
    \item Identificar patrones relevantes de corrupción mediante consultas SQL y modelado.
\end{itemize}

\section*{Metodología}

\subsection*{Diseño y Modelado}
\begin{itemize}
    \item Se creó un modelo de base de datos utilizando el diagrama ER para representar las relaciones entre personas, instituciones y empresas.
    \item Las tablas principales incluyen \texttt{person}, \texttt{institution}, \texttt{investigation}, \texttt{publication}, entre otras.
    \item Se incluyó un archivo SQL que genera la estructura completa de la base de datos.
\end{itemize}

\subsection*{Implementación}
\begin{itemize}
    \item Configuración del servidor PostgreSQL y creación de la base de datos \texttt{proyecto\_corrupcion}.
    \item Importación de datos reales en las tablas creadas.
\end{itemize}

\section*{Acceso al Repositorio}
Toda la documentación, scripts y resultados de este proyecto están disponibles en el repositorio de GitHub:

\begin{itemize}
    \item URL del repositorio: \href{https://github.com/CarlitoxVera/proyecto_corrupcion}{https://github.com/CarlitoxVera/proyecto_corrupcion}
    \item Usuario: \texttt{proyecto\_user}
    \item Contraseña: \texttt{proyecto2025}
\end{itemize}

\subsection*{Consultas Obligatorias}
Se desarrollaron 13 consultas que responden a los objetivos iniciales del proyecto:
\begin{enumerate}
    \item Fecha inicial desde que opera Luis Hermosilla.
    \item Principales colaboradores de Luis Hermosilla año a año.
    \item Medios que publican noticias sobre Luis Hermosilla.
    \item Ministerios relacionados con los actos de Luis Hermosilla.
    \item Ministros involucrados en acciones de Luis Hermosilla.
    \item Partidos políticos relacionados con Luis Hermosilla.
    \item Lazos familiares en el poder relacionados con Luis Hermosilla.
    \item Eventos como daños colaterales desde que se destapó el caso.
    \item Actos enunciados en el audio considerados delitos o faltas.
    \item Universidades de los involucrados en delitos realizados por Luis Hermosilla.
    \item Horas desde que se publicó el audio.
    \item Comunas con proyectos favorecidos por gestiones de Luis Hermosilla.
    \item Tamaño de información generado por categoría de contenido.
\end{enumerate}

\subsection*{Consultas Nuevas}
Además, se implementaron 10 consultas adicionales para analizar patrones emergentes:
\begin{enumerate}
    \item Personas involucradas en múltiples investigaciones.
    \item Rango de fechas de publicaciones relacionadas con Luis Hermosilla.
    \item Instituciones con mayor cantidad de investigaciones.
    \item Porcentaje de investigaciones relacionadas con tráfico de influencias.
    \item Proyectos gestionados en comunas específicas por Luis Hermosilla.
    \item Asesores legales que participan en múltiples investigaciones.
    \item Publicación más grande por tamaño.
    \item Personas con mayor número de relaciones familiares o laborales.
    \item Investigaciones relacionadas con evidencia clave.
    \item Empresas vinculadas a investigaciones de corrupción.
\end{enumerate}

\section*{Resultados y Análisis}

\subsection*{Datos Generados}
\begin{itemize}
    \item Tablas: 21 tablas principales fueron exportadas como archivos CSV.
    \item Consultas: Todas las consultas fueron implementadas y sus resultados documentados.
    \item Archivos SQL: Un archivo con todas las consultas implementadas y otro con sus resultados.
\end{itemize}

\subsection*{Hallazgos Principales}
\begin{itemize}
    \item Luis Hermosilla está vinculado directamente a 6 investigaciones.
    \item El Ministerio Público es la institución más involucrada, con 8 investigaciones relacionadas.
    \item La corrupción revelada abarca una red compleja de conexiones familiares, políticas y empresariales.
\end{itemize}

\section*{Conclusiones}
\begin{itemize}
    \item Este proyecto destaca cómo la base de datos puede ser utilizada para identificar patrones de corrupción.
    \item Las consultas desarrolladas ofrecen un marco sólido para futuros análisis y visualizaciones.
    \item Es posible extender este trabajo con integraciones a herramientas de análisis visual y reportes dinámicos.
\end{itemize}

\section*{Próximos Pasos}
\begin{itemize}
    \item Implementar un sistema de visualización interactivo.
    \item Realizar análisis predictivos sobre datos históricos.
    \item Mejorar la documentación técnica y generar reportes automáticos.
\end{itemize}

\section*{Referencias}
\begin{itemize}
    \item Documentación generada durante el proyecto.
    \item Archivos adjuntos: \texttt{consultas\_proyecto.sql}, \texttt{consultas\_nuevas\_resultados.csv}, entre otros.
\end{itemize}


\section*{Anexos}
\begin{itemize}
    \item Diagramas ER.
    \item Logs y resultados de consultas ejecutadas.
\end{itemize}

\end{document}

